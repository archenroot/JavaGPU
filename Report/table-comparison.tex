\subsection{Grades} \label{grades}

To rank our options, we will give, based on each criteria, a grade going from 0 to 10, 10 being the best and 0 the worst.

Time from update, adoption level and adequacy grade will go according to the following tables:

\begin{center}
\captionof{table}{Grade: Time from update}
\begin{tabular}{|c|c|c|c|c|}
	\hline
	< 3 months & < 6 months & < 1 years & < 2 years & > 2 years \\
	\hline
	10 & 8 & 5 & 3 & 0 \\
	\hline
\end{tabular}
\end{center}

\begin{center}
\captionof{table}{Grade: Adoption level, community activity and documentation}
\begin{tabular}{|p{.25\textwidth}|p{.25\textwidth}|p{.25\textwidth}|p{.25\textwidth}|}
	\hline
	Is well referenced, found many articles and is well documented & 
	Appears quite often in a google search, found some articles and has a fairly good documentation &
	Found some results in a google search and has a documentation &
	Didn't found much results in a google search, has a poor documentation \\
	\hline
	10 & 8 & 5 & 3 \\
	\hline
\end{tabular}
\end{center}

\begin{center}
\captionof{table}{Grade: Adequacy}
\begin{tabular}{|p{.25\textwidth}|p{.25\textwidth}|p{.25\textwidth}|p{.25\textwidth}|}
	\hline
	Fits perfectly our needs & 
	Fits partially our needs, or the solution is not aimed to particularly fits out needs &
	Doesn't fit our needs \\
	\hline
	10 & 3 & 0 \\
	\hline
\end{tabular}
\end{center}

\subsection{Grades table}

According to our pre-analysis (see section \ref{pre analysis}) and our grades (section \ref{grades}), each candidates has been evaluated and got a corresponding mark. Based on the marks, we will choose the candidates that have the highest ones.

\begin{longtable}{ | p{.20\textwidth} | p{.20\textwidth} | p{.20\textwidth} | p{.20\textwidth} | p{.10\textwidth} | }
  
  \hline
  
  & Time from update & Adoption level, community activity \& documenation & Adequacy & \textbf{Grade} \\
  
  \hline \hline
  
  \textbf{aparapi} &
  10 &
  10 &
  10 &
  \textbf{30} \\
  
  \hline \hline
  
  \textbf{jcuda} &
  10 &
  8 &
  10 &
  \textbf{28} \\
  
  \hline \hline
  
  \textbf{CUDA4J} &
  5 &
  10 &
  10 &
  \textbf{25} \\
  
  \hline \hline
 
  \textbf{PJ2} &
  10 &
  5 &
  10 &
  \textbf{25} \\
  
  \hline \hline

  \textbf{rootbeer} &
  5 &
  8 &
  10 & 
  \textbf{23} \\
  
  \hline \hline
  
  \textbf{java-gpu} &
  8 &
  5 &
  10 &
  \textbf{23} \\
  
  \hline \hline
  
  \textbf{JavaCL} &
  8 &
  5 &
  10 &
  \textbf{23} \\
  
  \hline \hline
  
  \textbf{lwjgl} &
  10 &
  10 &
  3 &
  \textbf{23} \\
  
  \hline \hline
  
  \textbf{jacc} &
  5 &
  5 &
  10 &
  \textbf{20} \\
  
  \hline \hline
  
  \textbf{jocl} &
  5 &
  5 &
  10 &
  20 \\
  
  \hline \hline
  
  \textbf{ScalaCL} &
  5 &
  5 &
  10 &
  20 \\
  
  \hline

\end{longtable}